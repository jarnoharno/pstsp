\documentclass[a4paper]{scrartcl}
\usepackage[finnish]{babel}
\usepackage[utf8]{inputenc}
\usepackage{verbatim}
\usepackage[numbers]{natbib}
\usepackage{hyperref}
\subject{Pretty Simple TSP}
\author{Jarno Leppänen}
\title{Määrittelydokumentti}
\date{20.1.2014}
\begin{document}
\maketitle

\section{Johdanto}
Pretty simple TSP (\texttt{pstsp}) on approksimatiivisen
algoritmin toteutus kauppamatkustajan ongelmalle.

\subsection{Teoriaa}
Symmetrisen verkon pienin virittävä puu on luonteva alaraja kauppamatkustajan
ongelman ratkaisulle, sillä minkä tahansa kaaren poistaminen ratkaisusta
tuottaa Hamiltonin polun, joka on samalla eräs verkon virittävä puu \cite{wi}.
Jos verkko on lisäksi täydellinen ja sen kaaret toteuttavat kolmioepäyhtälön,
eli suora reitti solmusta $a$ solmuun $b$ on aina lyhyempi
kuin reitti solmun $c$ kautta ($d_{ab} < d_{ac} + d_{cb}$), voidaan
laatia pienimpään virittävään puuhun perustuva algoritmi, jonka tuottama
ratkaisu on korkeintaan vakiokertoiminen optimaaliseen ratkaisuun verrattuna:
\begin{enumerate}
    \item Etsitään verkolle $G$ pienin virittävä puu $T$.
    \item Lisätään verkkoon $H$ puun $T$ kaaret ja jokaista puun $T$ kaarta
          $(u, v)$ vastaava kaari $(v, u)$.
    \item Etsitään jokaisen kaaren täsmälleen kerran läpikäyvä
          \emph{Eulerin polku} $C$
          verkossa $H$. Polun pituus on kaksi kertaa
          puun $T$ pituus.
    \item Laaditaan kaikki verkon $G$ solmut täsmälleen kerran läpikäyvä
          \emph{Hamiltonin sykli} kulkemalla polun $C$ kaaria pitkin ja
          hyppäämällä yli kaikki jo läpikäydyt solmut.
\end{enumerate}
Koska verkon kaaret toteuttavat kolmioepäyhtälön, on kohdassa 4 tuotettu sykli
lyhyempi kuin polku $C$. Algoritmi siis tuottaa kauppamatkustajan ongelmaan
ratkaisun, joka on korkeintaan kaksi kertaa optimaalista ratkaisua pidempi.

Ratkaisu voidaan myös tuottaa yhdistämällä kaarilla sellaiset 
pienimmän virittävän puun solmut, joiden asteluku on pariton luku ja
muodostamalla Eulerin polku syntyneessä verkossa.
\emph{Christofidesin algoritmissa}\cite{ch} yhdistävät kaaret valitaan
siten, että niiden painojen summa on mahdollisimman pieni. Voidaan
osoittaa, että näin saadun ratkaisun pituuden yläraja on $3/2$ kertaa 
optimiratkaisun pituus.

\section{Toteutettavat algoritmit}
\texttt{pstsp} koostuu kahdesta osasta.

\subsection{Lyhimpien polkujen haku kaikkien verkon solmujen välillä}
Ensimmäisessä osassa syötteenä oleva verkko käsitellään siten,
että esiteltyä algoritmia voidaan käyttää. Jos verkko
ei ole täydellinen tai toteuta kolmioepäyhtälöä, laaditaan uusi verkko
hakemalla lyhimmät polut kaikkien solmujen välillä ja asettamalla nämä
kaarien painoiksi. Kaikkien lyhimpien polkujen hakuun toteutetaan
Floyd-Warshallin algoritmi\cite{fw} ja harvoille verkoille paremmin
soveltuva Johnsonin algoritmi\cite{jo},
jos aikaa jää.

\subsection{Approksimatiivinen algoritmi kauppamatkustajan ongelmalle}
Toinen osa koostuu approksimatiivisen algoritmin toteutuksesta
kauppamatkustajan ongelmalle. Pienimmän virittävän puun etsintään
toteutetaan Primin algoritmi\cite{pr} ja Hamiltonin sykli etsitään
yksinkertaisella syvyyssuuntaisen haun muunnoksella. Jos aikaa jää,
toteutetaan Christofidesin algoritmi. Tässä tarvittavan 
parittoman asteluvun solmujen täydellisen
sovittamisen etsintään on olemassa polynomisessa ajassa toimiva
\emph{blossom-algoritmi}\cite{bl}, joka on tosin erittäin haastava
toteutettaa.

\section{Motivaatio}
Esitelty algoritmi soveltuu hyvin toteutettavaksi tietorakenteiden ja
algoritmien perusteiden opettelun jälkeen. Algoritmissa ja verkon
esikäsittelyssä hyödynnetään monia kurssilla esiteltyjä menetelmiä,
kuten lyhimpien polkujen ja pienimmän virittävän puun etsintää.

Käytännössä kauppamatkustajan ongelmaan on olemassa heuristiikkoja,
jotka löytävät nopeammin parempia ratkaisuja, kuin tässä työssä
käytettävät algoritmit.\cite{wi} Suurinkiin ongelmiin voidaan lisäksi löytää
eksakti ratkaisu kokonaislukuoptimoinnin menetelmillä.

\section{Laskennan vaativuus}
Työn ensimmäisen osa on toista osaa vaativampi laskennallisesti.
Floyd-Warshallin algoritmin aikavaativuus on
$\Theta(|V^3|)$ ja tilavaativuus $\Theta(|V^2|)$.
Binäärikekoa hyödyntävää Dijkstran algoritmia\cite{di} käyttävän Johnsonin
algoritmin pahimman tapauksen aikavaativuus on luokkaa
$\mathcal{O}((|V|+|E|)|V|\log|V|)$ ja tilavaativuus $\mathcal{O}(|V|)$. 

Binäärikekoa hyödyntävän Primin algoritmin pahimman tapauksen
aikavaativuus on luokkaa $\mathcal{O}((|V|+|E|)\log|V|)$ ja
tilavaativuus $\mathcal(|V|)$. Christofidesin algoritmin aikavaativuus
on laskennallisesti vaativan blossom-algoritmin vuoksi jopa
$\mathcal{O}(|V|^4)$.

\section{Ohjelman rakenne}
Työ koostuu C++-header-kirjastosta, jossa on toteutettu vaadittavat
algoritmit, sekä kirjastoa hyödyntävästä komentoriviltä käytettävästä
ohjelmasta. Syötteenä hyväksytään \texttt{TSPLIB}-formaatissa\cite{tsplib}
ja AI Odyssey -kurssilla\cite{ai} käytetyssä formaatissa olevia tiedostoja.
Ohjelma tunnistaa, tarvitseeko syötteenä oleva verkko käsittelyä ja
keskeyttää ajon, jos verkko sisältää negatiivisia syklejä.
Projektinhallintaan käytetään \texttt{cmake}-työkalua\cite{cmake},
yksikkötestaukseen \texttt{boost test}-kirjastoa\cite{gtest} ja
dokumentointiin \texttt{doxygen}-työkalua\cite{doxygen}.
Paperidokumenttien laatimiseen käytetään
\LaTeX-ladontajärjestelmää\cite{latex}.

\bibliographystyle{plainnat}
\bibliography{pstsp}
\end{document}
